\begin{problem}
庫侖定律和高斯定理互相推導過程
\begin{align*}
    (庫倫定律)\  & \va{E} = \frac{1}{4\pi\varepsilon_0} \frac{q}{r^2} \vu{r}                                       \\
    (高斯定理)\  & \oiint_{\symbb{S}} \va{E} \cdot \dif \va{S} = \frac{1}{\varepsilon_0} \sum_{i(\symbb{S}内)} q_i
\end{align*}
\end{problem}

\begin{solve}
    極座標下,對於半徑爲$r$的一球面面元$\dif A$,
    $$\dif A = (r\sin\theta\dif\varphi)(r\dif\theta)$$
    可定義極小立體角爲
    $$\dif\Omega = \frac{\dif A}{r^2} = \sin\theta\dif\theta\dif\varphi$$
    對於更加普遍的有向面元$\dif\va{S} = \dif S\vu{n}$而言, 可推廣立體角定義爲
    $$\dif\Omega  = \frac{(\vu{r}\cdot\vu{n})\dif S}{r^2} = \frac{\vu{r}\cdot\vu{n}}{|\vu{r}\cdot\vu{n}|}\sin\theta\dif\theta\dif\varphi$$
    由立體角的推廣定義,對於任一封閉曲面$\symbb{S}$所張之立體角也就有所定義了,當頂點在曲面内時, 對於曲面上的任一立體角元$\dif\Omega$,其$\vu{r}$和$\vu{n}$夾角總小於$\pi/2$, 即 $\dfrac{ \vu{r}\cdot\vu{n} }{ |\vu{r}\cdot\vu{n}| } \equiv 1$, 於是
    $$\oiint_{\symbb{S}} \dif\Omega = \oiint_{\symbb{S}} \sin\theta \dif\theta \dif\varphi = \int_0^\pi \sin\theta \dif\theta \int_0^{2\pi} \dif\varphi = [-cos\theta]_0^{\pi} (2\pi)  = 4\pi$$
    當頂點在曲面外時, 對於曲面上的任一立體角元$\dif\Omega$皆存在一個互爲相反數立體角元$\dif\Omega$, 使得積分結果爲$0$
    $$\oiint_{\symbb{S}} \dif\Omega = 0$$
    \begin{itemize}
        \item[\textbf{1)}] \textbf{庫倫定律 $\rightarrow$ 高斯定理}

              數學中定義積分式:
              $$\iint_{\Sigma} \va{A} \cdot \vu{n} \dif S = \iint_{\Sigma} \va{A} \cdot \dif\va{S}$$
              爲向量場$\va{A}$通過曲面$\Sigma$向着指定側的\uwave{通量}, 其中$\vu{n}$ 是面元的單位法向量.
              對於電場$\va{E}$, 自然可定義\uwave{電通量}
              $$\Phi_{E} = \iint_{\Sigma} \va{E} \cdot \vu{n} \dif S = \iint_{\Sigma} \va{E} \cdot \dif\va{S}$$
              今假設空間中存在一封閉曲面$\symbb{S}$名曰\uwave{高斯面}. 則可依單個點電荷$q$與$\symbb{S}$的位置關係而對$\symbb{S}$向外側的電通量進行分類討論:
              \columnseprule=0.4pt
              \begin{multicols}{2}
                  \begin{itemize}
                      \item[a)] $q$在$\symbb{S}$内的情況
                            \begin{align*}
                                \Phi_E & = \oiint_{\symbb{S}} \va{E} \cdot \dif \va{S}                                          \\
                                       & = \oiint_{\symbb{S}} \frac{1}{4\pi\varepsilon_0} \frac{q}{r^2} \vu{r} \cdot \dif\va{S} \\
                                       & = \oiint_{\symbb{S}} \frac{q}{4\pi\varepsilon_0} \dif\Omega                            \\
                                       & = \frac{q}{\varepsilon_0}
                            \end{align*}

                      \item[b)] $q$在$\symbb{S}$外的情況
                            \begin{align*}
                                \Phi_E & = \oiint_{\symbb{S}} \va{E} \cdot \dif \va{S}                                          \\
                                       & = \oiint_{\symbb{S}} \frac{1}{4\pi\varepsilon_0} \frac{q}{r^2} \vu{r} \cdot \dif\va{S} \\
                                       & = \oiint_{\symbb{S}} \frac{q}{4\pi\varepsilon_0} \dif\Omega                            \\
                                       & = 0
                            \end{align*}
                  \end{itemize}
              \end{multicols}

              因爲任何電荷系的點電荷皆可分爲$\symbb{S}$內和$\symbb{S}$外兩個部分, 由a)、b)的結論和場強疊加原理得
              \begin{align*}
                  \Phi_E & = \oiint_{\symbb{S}}\left(\sum_{i(\symbb{S}内)}\va{E}_i + \sum_{j(\symbb{S}外)}\va{E}_j\right)\cdot \dif \va{S}                       \\
                         & = \sum_{i(\symbb{S}内)}\oiint_{\symbb{S}}\va{E}_i\cdot \dif \va{S} + \sum_{j(\symbb{S}外)}\oiint_{\symbb{S}}\va{E}_j\cdot \dif \va{S} \\
                         & = \sum_{i(\symbb{S}内)}\frac{q_i}{\varepsilon_0} + \sum_{j(\symbb{S}外)} 0                                                            \\
                         & = \frac{1}{\varepsilon_0}\sum_{i(\symbb{S}内)} q_i                                                                                    \\
              \end{align*}


        \item[\textbf{2)}] \textbf{高斯定理 $\rightarrow$ 庫倫定律}

              設一點電荷$q$, 以其爲球心, $r$爲半徑假設球形高斯面, 則有
              $$\oiint_{\symbb{S}} \va{E} \cdot \dif\va{S} = \oiint_{\symbb{S}} \va{E} \cdot \vu{n} \dif S = \frac{q}{\varepsilon_0}$$
              因爲球面上$\vu{n} = \vu{r}$且$\va{E}$和$\vu{r}$處處平行, 場強大小在球面上處處相等, 故
              $$\va{E} \cdot \vu{n} \oiint_{\symbb{S}} \dif S = \va{E} \cdot \vu{r} \oiint_{\symbb{S}} \dif S = \frac{q}{\varepsilon_0}$$
              又因爲$\vu{r}^2 = 1$, $\vu{r}$自反, 所以
              \begin{equation*}
                  \va{E} = \frac{\vu{r}}{4\pi r^2}\frac{q}{\varepsilon_0} = \frac{1}{4\pi\varepsilon_0}\frac{q}{r^2}\vu{r} \qedhere
              \end{equation*}
    \end{itemize}
\end{solve}