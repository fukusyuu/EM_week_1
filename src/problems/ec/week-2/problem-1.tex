%!ORDER = 1
\begin{problem}
次の関数をラプラス逆変換せよ
$$\frac{2s+3}{(s+3)^2(s+4)}$$
\end{problem}

\newcommand*{\lapr}{\mscrL^{-1}}
\begin{solve}
    \pair{
        与式は
        \begin{equation}
            \frac{2s+3}{(s+3)^2(s+4)} = \frac{A}{(s+3)^2} + \frac{B}{s+3} + \frac{C}{s+4}
        \end{equation}
        とする. (1)式の両辺には$(s+3)^2(s+4)$をかけると,
        \begin{equation}
            2s+3 = A(s+4) + B(s+3)(s+4) +C(s+3)^2
        \end{equation}
        になる. そこで, $s = -3$とすると, $A = -3$. $s = -4$とすると, $C = -5$.
        $A$と$C$は(2)式に代入すると,
        \begin{align*}
            2s+3 & = -3s - 12 + Bs^2 + 7Bs + 12B + Cs^2 + 6Cs + 9C \\
                 & = (B+C)s^2 + (7B + 6C -3)s + (12B + 9C -12)
        \end{align*}
        両辺の係数を比較すると, $B+C = 0$, $7B + 6C -3=2$, $12B + 9C -12=3$. それで,
        $$
            \begin{cases}
                A = -3 \\
                B = 5  \\
                C = -5
            \end{cases}
        $$
        (1)式には$A$,$B$と$C$の値を代入すると,
        $$
            \frac{2s+3}{(s+3)^2(s+4)} = \frac{-3}{(s+3)^2} + \frac{5}{s+3} + \frac{-5}{s+4}
        $$}{
        与式を部分分数に分解する
    }
    \pair{
        ラプラス変換表より,
        \begin{align*}
            \lapr[与式] & = -3\lapr\left[\frac{1}{(s+3)^2}\right] +5\lapr\left[\frac{1}{s+3}\right] -5\lapr\left[\frac{1}{s+4}\right] \\
                        & =  -3t\ce^{-3t} + 5\ce^{-3t} -5\ce^{-4t}\qedhere
        \end{align*}
    }{
        ラプラス変換表及びラプラス逆変換の性質を利用し、結果を求める.
        \begin{align*}
             & \mscrL\left[\ce^{at}\right] = \frac{1}{s-a}                      \\
             & \mscrL\left[t^m\ce^{at}\right] = \frac{\Gamma(m+1)}{(s-a)^{m+1}}
        \end{align*}
    }
\end{solve}