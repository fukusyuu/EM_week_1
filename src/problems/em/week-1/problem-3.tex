% !ORDER = 3

\begin{problem}
相互證明庫侖定律和高斯定理
\begin{align*}
    (庫侖定律)\  & \vE = \frac{1}{4\pi\eo} \frac{q}{r^2} \vr                       \\
    (高斯定理)\  & \oiint_{\BbbS} \vE \cdot \dvS = \frac{1}{\eo} \sum_{i(\iS)} q_i
\end{align*}
\end{problem}

\begin{solve}
    極座標下,對於半徑爲$r$的一球面面元$\dA$有:
    $$\dA = (r\sin\theta\dph)(r\dth)$$
    可定義極小立體角爲:
    $$\dO = \frac{\dA}{r^2} = \sin\theta\dth\dph$$
    一般地, 對於有向面元$\dvS = \dS\vn$, 可推廣立體角定義爲:
    $$\dO  = \frac{(\vr\cdot\vn)\dS}{r^2} = \frac{\vr\cdot\vn}{|\vr\cdot\vn|}\sin\theta\dth\dph$$
    是以, 對於任一封閉曲面$\BbbS$, 當頂點在曲面内時, 對於曲面上的任一立體角元$\dO$,總有$\langle \vr,\vn \rangle < \dfrac{\pi}{2}$, 即 $\dfrac{ \vr\cdot\vn }{ |\vr\cdot\vn| } \equiv 1$, 於是
    $$\oiint_{\BbbS} \dO = \oiint_{\BbbS} \sin\theta \dth \dph = \int_0^\pi \sin\theta \dth \int_0^{2\pi} \dph = [-cos\theta]_0^{\pi} (2\pi)  = 4\pi$$
    當頂點在曲面外時, 對於頂點所張的任一$\dO$皆存在一個互爲相反數的$-\dO$, 使得積分結果爲$0$:
    $$\oiint_{\BbbS} \dO = 0$$
    \begin{itemize}
        \item[\textbf{1)}] \textbf{庫侖定律 $\rightarrow$ 高斯定理}

              數學中定義積分式:
              $$\iint_{\Sigma} \vA \cdot \vn \dS = \iint_{\Sigma} \vA \cdot \dvS$$
              爲向量場$\vA$通過曲面$\Sigma$向着指定側的\uwave{通量}, 其中$\vn$ 是面元的單位法向量.
              對於電場$\vE$, 可定義\uwave{電通量}:
              $$\Phi_{E} = \iint_{\Sigma} \vE \cdot \vn \dS = \iint_{\Sigma} \vE \cdot \dvS$$
              今假設空間中存在一封閉曲面$\BbbS$曰\uwave{高斯面}, 則可依點電荷$q$與$\BbbS$的位置關係而對$\BbbS$向外側的電通量進行分類討論:
              \columnseprule=0.4pt
              \begin{multicols}{2}
                  \begin{itemize}
                      \item[a)] $q$在$\BbbS$内的情況
                            \begin{align*}
                                \Phi_E & = \oiint_{\BbbS} \vE \cdot \dvS                                 \\
                                       & = \oiint_{\BbbS} \frac{1}{4\pi\eo} \frac{q}{r^2} \vr \cdot \dvS \\
                                       & = \oiint_{\BbbS} \frac{q}{4\pi\eo} \dO                          \\
                                       & = \frac{q}{\eo}
                            \end{align*}

                      \item[b)] $q$在$\BbbS$外的情況
                            \begin{align*}
                                \Phi_E & = \oiint_{\BbbS} \vE \cdot \dvS                                 \\
                                       & = \oiint_{\BbbS} \frac{1}{4\pi\eo} \frac{q}{r^2} \vr \cdot \dvS \\
                                       & = \oiint_{\BbbS} \frac{q}{4\pi\eo} \dO                          \\
                                       & = 0
                            \end{align*}
                  \end{itemize}
              \end{multicols}

              因爲任何電荷系的點電荷皆可分爲$\BbbS$內和$\BbbS$外兩個部分, 由a)、b)的結論和場強疊加原理得
              \begin{align*}
                  \Phi_E & = \oiint_{\BbbS}\left(\sum_{i(\iS)}\vE_i + \sum_{j(\oS)}\vE_j\right)\cdot \dvS            \\
                         & = \sum_{i(\iS)}\oiint_{\BbbS}\vE_i\cdot \dvS + \sum_{j(\oS)}\oiint_{\BbbS}\vE_j\cdot \dvS \\
                         & = \sum_{i(\iS)}\frac{q_i}{\eo} + \sum_{j(\oS)} 0                                          \\
                         & = \frac{1}{\eo}\sum_{i(\iS)} q_i                                                          \\
              \end{align*}


        \item[\textbf{2)}] \textbf{高斯定理 $\rightarrow$ 庫侖定律}

              對於點電荷$q$, 以其爲球心, $r$爲半徑設一球形高斯面$\BbbS$, 則有
              $$\oiint_{\BbbS} \vE \cdot \dvS = \oiint_{\BbbS} \vE \cdot \vn \dS = \frac{q}{\eo}$$
              在$\BbbS$上$\vn \equiv \vr$且$\vE \parallel \vr$, 場強大小在$\BbbS$上處處相等, 是故
              $$\vE \cdot \vr \oiint_{\BbbS} \dS = \vE \cdot \vn \oiint_{\BbbS} \dS = \frac{q}{\eo}$$
              $\vr^2 = 1$, 故$\vr$自反, 於是
              \begin{equation*}
                  \vE = \frac{\vr}{4\pi r^2}\frac{q}{\eo} = \frac{1}{4\pi\eo}\frac{q}{r^2}\vr \qedhere
              \end{equation*}
    \end{itemize}
\end{solve}