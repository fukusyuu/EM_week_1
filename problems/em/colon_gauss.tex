\begin{problem}
庫侖定律和高斯定理互相推導過程
\begin{align*}
    \text{庫倫定律:} & \mathbf{E} = \frac{1}{4\pi\varepsilon_0} \frac{q}{r^2} \hat{\mathbf{r}}                                                     \\
    \text{高斯定理:} & \Phi_{E} = \oiint_{\mathbb{S}} \mathbf{E} \cdot \dif \mathbf{S} = \frac{1}{\varepsilon_0} \sum_{i(\mathbb{S}\text{内})} q_i
\end{align*}

\end{problem}

\begin{solve}
    極座標下,對於半徑爲$r$的一球面面元$\dif A$,
    $$\dif A = (r\sin\theta\dif\varphi)(r\dif\theta)$$
    可定義極小立體角爲
    $$\dif\Omega = \frac{\dif A}{r^2} = \sin\theta\dif\theta\dif\varphi$$
    對於更加普遍的有向面元$\dif\va{S} = \dif S\vu{n}$而言, 可推廣立體角定義爲
    $$\dif\Omega  = \frac{(\vu{r}\cdot\vu{n})\dif S}{r^2} = \frac{\vu{r}\cdot\vu{n}}{\left\|\vu{r}\cdot\vu{n}\right\|}\sin\theta\dif\theta\dif\varphi$$
    由立體角的推廣定義,對於任一封閉曲面$\mathbb{S}$所張之立體角也就有所定義了,當頂點在曲面内時, 對於曲面上的任一立體角元$\dif\Omega$,其$\vu{r}$和$\vu{n}$夾角總小於$\pi/2$, 即 $(\vu{r}\cdot\vu{n})/{\left\|\vu{r}\cdot\vu{n}\right\|} \equiv 1$
    $$
        \oiint_\mathbb{S} \dif\Omega = \oiint_\mathbb{S} \sin\theta \dif\theta \dif\varphi = \int_0^\pi \sin\theta \dif\theta \int_0^{2\pi} \dif\varphi = [-cos\theta]_0^{\pi} (2\pi)  = 4\pi
    $$
    當頂點在曲面外時, 對於曲面上的任一立體角元$\dif\Omega$皆存在一個互爲相反數立體角元$\dif\Omega$, 使得積分結果爲$0$
    $$
        \oiint_\mathbb{S} \dif\Omega = 0
    $$
    \begin{itemize}
        \item[1)] 庫倫定律 $\rightarrow$ 高斯定理\\
              電通量$\Phi_E$乃電場在高斯面各處的法向分量的總和.

              \columnseprule=0.4pt
              \begin{multicols}{2}
                  \begin{itemize}
                      \item[a)] $q$在高斯面内的情況
                            \begin{align*}
                                \Phi_E & = \oiint_{\mathbb{S}} \mathbf{E} \cdot \dif \mathbf{S}                                            \\
                                       & = \oiint_{\mathbb{S}} \frac{1}{4\pi\varepsilon_0} \frac{q}{r^2} \hat{\mathbf{r}} \cdot \dif\va{S} \\
                                       & = \oiint_{\mathbb{S}} \frac{q}{4\pi\varepsilon_0} \dif\Omega                                      \\
                                       & = \frac{q}{\varepsilon_0}
                            \end{align*}

                      \item[b)] $q$在高斯面外的情況
                            \begin{align*}
                                \Phi_E & = \oiint_{\mathbb{S}} \mathbf{E} \cdot \dif \mathbf{S}                                            \\
                                       & = \oiint_{\mathbb{S}} \frac{1}{4\pi\varepsilon_0} \frac{q}{r^2} \hat{\mathbf{r}} \cdot \dif\va{S} \\
                                       & = \oiint_{\mathbb{S}} \frac{q}{4\pi\varepsilon_0} \dif\Omega                                      \\
                                       & = 0
                            \end{align*}
                  \end{itemize}
              \end{multicols}


              \begin{itemize}
                  \item[c)] 數個點電荷組成電荷系的情況

                        電荷系可分爲面内和面外兩個部分, 由a)、b)的結論和場強疊加原理得
                        \begin{align*}
                            \Phi_E & = \oiint_{\mathbb{S}}\left(\sum_{i(\mathbb{S}\text{内})}\mathbf{E}_i + \sum_{j(\mathbb{S}\text{外})}\mathbf{E}_j\right)\cdot \dif \mathbf{S}                            \\
                                   & = \sum_{i(\mathbb{S}\text{内})}\oiint_{\mathbb{S}}\mathbf{E}_i\cdot \dif \mathbf{S} + \sum_{j(\mathbb{S}\text{外})}\oiint_{\mathbb{S}}\mathbf{E}_j\cdot \dif \mathbf{S} \\
                                   & = \sum_{i(\mathbb{S}\text{内})}\frac{q_i}{\varepsilon_0} + \sum_{j(\mathbb{S}\text{外})} 0                                                                              \\
                                   & = \frac{1}{\varepsilon_0}\sum_{i(\mathbb{S}\text{内})} q_i                                                                                                              \\
                        \end{align*}
              \end{itemize}


        \item[2)] 高斯定理 $\rightarrow$ 庫倫定律\\
              設一點電荷$q$, 以其爲球心, $r$爲半徑假設球形高斯面, 則有
              $$\oiint_{\mathbb{S}} \va{E} \cdot \dif\va{S} = \frac{q}{\varepsilon_0}$$
              $$\oiint_{\mathbb{S}} \va{E} \cdot \vu{r} \dif S = \frac{q}{\varepsilon_0}$$
              因爲$\va{E}$和$\vu{r}$處處平行, 場強大小在球面上處處相等, 故
              $$\va{E} \cdot \vu{r} \oiint_{\mathbb{S}} \dif S = \frac{q}{\varepsilon_0}$$
              又因爲$\vu{r}^2 = 1$, $\vu{r}$自反, 所以
              \begin{equation*}
                  \va{E} = \frac{\vu{r}}{4\pi r^2}\frac{q}{\varepsilon_0} = \frac{1}{4\pi\varepsilon_0}\frac{q}{r^2}\vu{r} \qedhere
              \end{equation*}
    \end{itemize}
\end{solve}